\documentclass[12pt,a4paper,final]{article}
\makeatletter
\def\input@path{{../setup/}{.}}
\makeatother
\usepackage{cph-document}
\setlength{\parindent}{0cm}     % Default is 15pt.
\setlength{\parskip}{1em plus4mm minus3mm}


\makeindex

\title{Functional Programming}
\author{Anders Kalhauge and Jens Egholm Petersen}

\date{Spring 2018}

\begin{document}
\maketitle

The goal of this course is to introduce the students to functional programming. Programming functional requires another mind-set than procedural and especially object-oriented programming. The students will use functional programming in practice. Also the students will be able to decide where to use which paradigm.

After living in the dark for several decades, functional programming is having a revival. The demand for parallelism, due to the change in CPU architecture with more and more cores, are challenging languages as C\# and Java. Functional programming handles parallel programming very elegant and efficient. The newest versions of object oriented languages introduce functional programming constructs to support parallelism. Knowing functional programming will be a demand very soon.

\section*{Course content}

The course will look at functional programming using various programming languages.

At the end of the course the students:

\begin{itemize}
	\item Have a general knowledge about the functional paradigm
	\item Knows the building blocks of a functional programming language
	\item Knows how to support parallelism using a functional language
	\item Knows the overall constructs in a functional language
  \item Knows where to find additional information
\end{itemize}

At the end of the course the students can:

\begin{itemize}
	\item Write basic web applications using Elm
	\item Understand and write programs written in LISP
  \item Understand and write simple programs written in Haskell
\end{itemize}

The students will in groups make four major assignments in central topics of the curriculum. The solution of one of these will form the basis for the exam.
Also the students will do eight smaller assignments.

\section*{Exam}

The exam is oral.
The student will prepare a (app. ten minutes) presentation
of the solution of one of the major assignments.
Further discussions will be based on the presentation,
but can include all aspects of the curriculum.


\section*{Admission requisite}

In order to be approved for the exam:
\begin{itemize}
	\item All four major assignments must be handed in
	\item At least 80 study points must be obtained
\end{itemize}


\section*{Study points}

\begin{itemize}
	\item Hand in of major assignments (15 per assignment): 60
	\item Hand in of minor assignments (10 per assignment): 40
\end{itemize}


\end{document}
